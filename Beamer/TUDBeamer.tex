% Copyright 2007 by Till Tantau
%
% This file may be distributed and/or modified
%
% 1. under the LaTeX Project Public License and/or
% 2. under the GNU Public License.
%
% See the file doc/licenses/LICENSE for more details.



\documentclass{beamer}

%
% DO NOT USE THIS FILE AS A TEMPLATE FOR YOUR OWN TALKS�!!
%
% Use a file in the directory solutions instead.
% They are much better suited.
%


% Setup appearance:

\usetheme{Darmstadt}
\usefonttheme[onlylarge]{structurebold}
\setbeamerfont*{frametitle}{size=\normalsize,series=\bfseries}
\setbeamertemplate{navigation symbols}{}


% Standard packages

\usepackage[english]{babel}
\usepackage[latin1]{inputenc}
\usepackage{times}
\usepackage[T1]{fontenc}

% Setup TikZ
\usepackage{xcolor}
\usepackage{tikz}
\usetikzlibrary{arrows}
\tikzstyle{block}=[draw opacity=0.7,line width=1.4cm]


% Author, Title, etc.

\title[Distributed $3$D-Print Driver] 
{%
 Distributed \textbf{$3$}D-Print Driver
 %
}

\author[Guide]{
	\textbf{Presenter}: Suman~Bidarahalli \\
	\textbf{Supervisor}: \\
	Prof. Dr. Arjan~Kuijper \\  
	Alan~Brunton	\\ 
	Marco~Dennst{\"a}dt 
}
\institute[TU Darmstadt]
{
  Technische Universit�t Darmstadt, Germany
}

\date[ Master Thesis Presentation 2017]
{Master Thesis Presentation, February 2017}



% The main document

\begin{document}

\begin{frame}
  \titlepage
\end{frame}

\begin{frame}{Outline}
  \tableofcontents
\end{frame}


\section{Introduction}

\subsection{State of the art $3$D Print Driver}

\begin{frame}{Motivation}

\begin{columns}
    \begin{column}{0.25\textwidth}
    \centering
    \begin{figure}[!ht]
			\includegraphics[width=0.90\textwidth]{PrinterI}
			\caption{State of the art $3$D Printers}
			\label{Fig:State of the art $3$D Printers}
		\end{figure}
    \end{column}
    \begin{column}{0.75\textwidth}
    \centering
    \begin{itemize}
			\item Today's $3$D Printers allow for multi-material high resolution prints
			\item Multiple objects in varying size can be printed at a go
			\item Voxel-level material assignment enables to reproduce high-fidelity appearance properties
			\item Larger print objects are composed of huge number of voxels
		\end{itemize} 
    \end{column}
  \end{columns}
\end{frame}

\subsection{Problem Statement}

\begin{frame}[t]{Why do we need a Distributed $3$D Print Driver?}
\begin{columns}
    \begin{column}{0.45\textwidth}
    \centering
    \begin{figure}[!ht]
			\includegraphics[width=0.90\textwidth]{SillyImage}
			\label{Fig:MythI}
		\end{figure}
    \end{column}
    \begin{column}{0.55\textwidth}
    \centering
    \begin{itemize}
    \item Bigger prints require large amount of computational effort 
    \item Single workstation is limited in terms of resources  
		\item Existing Cuttlefish application designed to run on single machine 
    \item Distributed computing is one way of achieving better performance for large computations 
		\end{itemize} 
    \end{column}
  \end{columns}
\end{frame}

\section{Solution}

\subsection{What is a Distributed system? }
\begin{frame} {Infrastructure for distribution}
\begin{columns}
  \begin{column}{0.50\textwidth}
  \centering
	\begin{figure}[!ht]
			\includegraphics[width=0.90\textwidth]{DistributeIII.PNG}
			\caption{Distributed System}
			\label{Fig:DS}
		\end{figure}
	\end{column}
	\begin{column}{0.50\textwidth}
		 A \textbf{\textit{Distributed System}} is a cluster of AS where in each AS is connected to other by a network and primarily communicate via message passing
	\end{column}
\end{columns}
\end{frame}

\begin{frame}{Currently used Distributed System}
\begin{columns}
  \begin{column}{0.50\textwidth}
  \centering
    \begin{figure}[!ht]
			\includegraphics[width=0.90\textwidth]{ArchitectureI.png}
			\caption{Distributed System Architecture}
			\label{Fig:DCArchitecture}
		\end{figure}
	\end{column}
	\begin{column}{0.50\textwidth}
	\begin{itemize}
	\item The distributed system used is a cluster of heterogeneous systems 
	\item The nodes are connected via local(Fraunhofer) network
	\item All the nodes also have access to a shared network file system
	\end{itemize}	
	\end{column}
\end{columns}
\end{frame}

\begin{frame}{Master-Slave Paradigm}
\begin{columns}
  \begin{column}{0.30\textwidth}
  \centering
    \begin{figure}[!ht]
			\includegraphics[width=0.90\textwidth]{TaskSubmission.PNG}
			\caption{Step 1:Task Submission}
			\label{Fig:TaskSubmission}
		\end{figure}
	\end{column}
	\begin{column}{0.30\textwidth}
		\begin{figure}[!ht]
			\includegraphics[width=0.90\textwidth]{TaskDistribution.PNG}
			\caption{Step 2: Sub-Task Computation}
			\label{Fig:TaskDistribution}
		\end{figure}
	\end{column}
		\begin{column}{0.30\textwidth}
		\begin{figure}[!ht]
			\includegraphics[width=0.90\textwidth]{CollectionAndMerging.PNG}
			\caption{Step 3: Partial Result Collection and Merging}
			\label{Fig:CollectionAndMerging}
		\end{figure}
	\end{column}
\end{columns}
\end{frame}

%\begin{frame}{Sub-Task Computation \& Partial Results Reporting}
%\begin{columns}
  %\begin{column}{0.50\textwidth}
  %\centering
    %\begin{figure}[!ht]
			%\includegraphics[width=0.90\textwidth]{ArchitectureIII}
			%\caption{Sub-Task Computation}
			%\label{Fig:ArchitectureIII}
		%\end{figure}
	%\end{column}
	%\begin{column}{0.50\textwidth}
	%\begin{itemize}
	%\item The slave nodes perform the computation locally
	%\item The output of the computation performed by the slaves is then reported back to the master node
	%\item The master node performs final computation using the received partial output from the slaves to provide the user with the final output
	%\end{itemize}	
	%\end{column}
%\end{columns}
%\end{frame}

\subsection{How is the distribution done?}
\begin{frame}{Different possibilities of distribution}
\begin{columns}
   \begin{column}{0.50\textwidth}
    \centering
    \begin{figure}[!ht]
			\includegraphics[width=0.90\textwidth]{DistributeI}
			\caption{Distribution of one large print object amongst the workstations }
			\label{Fig:DistributeI}
		\end{figure}
   \end{column}
	\begin{column}{0.50\textwidth}
    \centering
    %\begin{itemize}
		 %\item Distribution of one large print object amongst the workstations 
		 %\item Distribution of multiple print objects amongst the workstations
		 %\end{itemize} 
			\begin{figure}[!ht]
			\includegraphics[width=0.90\textwidth]{DistributeII}
			\caption{Distribution of multiple print objects amongst the workstations}
			\label{Fig:DistributeII}
		\end{figure}
    \end{column}
  \end{columns}
  \end{frame}

%\begin{frame}{Chosen distribution}
    %\begin{itemize}
		%\item Multiple print objects will be distributed amongst the workstations i.e. each workstation performs computation on one or more whole inputs
		%\item The distribution of the workload and the collection of partial results is done through MPI(Message Passing Interface)
		%\end{itemize}
%\end{frame}

\subsection{Distributed Cuttlefish Application}
\begin{frame}{Current Component Architecture}
\begin{block}{Streaming Architecture}
	\begin{figure}[!ht]
		\includegraphics[width=0.9\textwidth]{StreamingArchitectureI.PNG}
		\caption{Cuttlefish Streaming Architecture}
		\label{Fig:CuttlefishPipelineFigure}
	\end{figure}
\end{block}
		%\item Cuttlefish Streaming architecture enables to perform the computation serially in a chunk-wise manner for each print object 
	\textbf{Goal}: To \textbf{retain the architecture} and is achieved by providing distinguished components to handle
	\begin{itemize} 
	\item  Distribution of sub-tasks 
	\item  Distribution of partial results
	\item  Collection of the partial results
	\end{itemize} 
\end{frame}


\begin{frame}{Classification of Prototypes}
\begin{block}{Distributed Cuttlefish Prototypes}
	\begin{figure}[!ht]
		\includegraphics[width=0.9\textwidth]{Prototypes.PNG}
		\label{Fig:StreamingArchitectureI}
	\end{figure}
\end{block}
\end{frame}

\begin{frame}{Modified Component Architecture- Prototype I}
\begin{block}{Streaming Architecture- Master Printing Software }
	\begin{figure}[!ht]
		\includegraphics[width=0.9\textwidth]{MSPSPI.png}
		\label{Fig:MasterPSPl}
	\end{figure}
\end{block}
	\begin{block}{Streaming Architecture- Slave Printing Software }
	\begin{figure}[!ht]
		\includegraphics[width=0.8\textwidth]{SlavePSPI.png}
		\label{Fig:SlavePSPI}
	\end{figure}
\end{block}
\end{frame}

\begin{frame}{Modified Component Architecture- Prototype II}
\begin{block}{Streaming Architecture- Master Printing Software }
	\begin{figure}[!ht]
		\includegraphics[width=0.9\textwidth]{MSPSPII.png}
		\label{Fig:MasterPSPIl}
	\end{figure}
\end{block}
	\begin{block}{Streaming Architecture- Slave Printing Software }
	\begin{figure}[!ht]
		\includegraphics[width=0.8\textwidth]{SlavePSPII.png}
		\label{Fig:SlavePSPII}
	\end{figure}
\end{block}
\end{frame}

\begin{frame}{Distribution of Sub-Tasks Using Cost Function}
\begin{itemize}
	\item For each object \textit{i}, compute \begin{math}b_{min}^i\end{math} and \begin{math}b_{max}^i\end{math}
	\item Calculate the width(w), height(h),and length(l)
	\begin{itemize}
		\item w=\begin{math}(b_{max}^i).y-(b_{min}^i).y\end{math} 
		\item h=\begin{math}(b_{max}^i).z-(b_{min}^i).z\end{math}
		\item l=\begin{math}(b_{max}^i).x-(b_{min}^i).x\end{math}
	\end{itemize}
	\item Volume of the object \begin{math}V^i= w*h*l\end{math}
	\item Total Volume(k objects)=\begin{math}\sum\limits_{i=1}^{k}{V^i}\end{math}
	\item Threshold (for cluster size n-1)=\begin{math}(\sum\limits_{i=1}^{k}{V^i})/(n-1)\end{math}
\end{itemize}
\end{frame}		

\begin{frame}{Producer-Consumer}
	\begin{figure}
		\includegraphics[width=0.9\textwidth]{ProducerConsumer.PNG}
	\end{figure}	
\end{frame}

\begin{frame}{Slave Reporter Component}
	\begin{figure}
		\includegraphics[width=0.50\textwidth]{SlaveReporter.pdf}
		\label{Fig:SlaveReporter}
	\end{figure}	
\end{frame}

\begin{frame}{Master Merger Component}
	\begin{figure}
		\includegraphics[width=0.55\textwidth]{MasterMerger.pdf}
		\label{Fig:MasterMerger}
	\end{figure}	
\end{frame}

\begin{frame}{Abstraction Layer}
	\begin{figure}
		\includegraphics[width=0.9\textwidth]{AbstractionLayer.PNG}
	\end{figure}	
\end{frame}

\begin{frame}{Additional Tasks}
\begin{block}{Serialization And Deserialization}
	\begin{figure}[!ht]
		\includegraphics[width=0.7\textwidth]{Ser-Deser.PNG}
		\label{Fig:Ser-Deser}
	\end{figure}
\end{block}
	\begin{block}{Compression And Decompression}
	\begin{figure}[!ht]
		\includegraphics[width=0.7\textwidth]{CompDecomp.PNG}
		\label{Fig:CompDecomp}
	\end{figure}
\end{block}
\end{frame}



\section{Results}
\subsection{Distributed Cuttlefish vs Non-Distributed Cuttlefish}
\begin{frame}
\begin{figure}[!ht]
		\includegraphics[width=0.9\textwidth]{SUPIPIIvsNumMod.pdf}
		\label{Fig:SUPIPIIvsNumMod}
\end{figure}
\end{frame}

\subsection{Prototype I vs Prototype II}
\begin{frame}
\begin{figure}[!ht]
		\includegraphics[width=0.9\textwidth]{SpeedupProtoIvsPrototII.pdf}
		\label{Fig:SpeedupProtoIvsPrototII}
\end{figure}
\end{frame}

\subsection{Prototype II Increase in Speed-Up}
\begin{frame}
\begin{figure}[!ht]
		\includegraphics[width=0.9\textwidth]{IncSpeedUpProtoIIVsCS.pdf}
		\label{Fig:IncSpeedUpProtoIIVsCS}
\end{figure}
\end{frame}

\subsection{Profiling Prototype I}
\begin{frame}
\begin{figure}[!ht]
		\includegraphics[width=0.9\textwidth]{DCPIvsCS.pdf}
		\label{Fig:DCPIvsCS}
\end{figure}
\end{frame}

\subsection{Profiling Prototype II}
\begin{frame}
\begin{figure}[!ht]
		\includegraphics[width=0.9\textwidth]{DCPIIvsSize.pdf}
		\label{Fig:DCPIIvsSize}
\end{figure}
\end{frame}

\subsection{Prototype II: Design I vs Design II}
\begin{frame}
\begin{figure}[!ht]
		\includegraphics[width=0.9\textwidth]{SUDIDIIvsNumMod.pdf}
		\label{Fig:SUDIDIIvsNumMod}
\end{figure}
\end{frame}


\section*{Conclusion}
\begin{frame}
  \frametitle<presentation>{Conclusion}
	\begin{itemize}
	\item Distributed Computing is useful to perform large-scale computation
	\item Super-linear speed-up can be achieved through distributed cuttlefish
	\item Prototype II scales better than Prototype I
	\item Communication between the nodes affects the performance of Prototype II
	\item Serialization and Deserialization affects the performance of Prototype I 
	\item Component Configuration has significant impact on the performance
	\end{itemize} 
\end{frame}


\section*{Future Work}
\begin{frame}
  \frametitle<presentation>{Future Work}
	\begin{itemize}
	\item Develop new approaches for cost function calculation 
	\item Implementation of new communication mechanism 
	\item Change in topology of the distributed system as well hardware acceleration
	\item Replacing cluster computing with cloud computing
	\end{itemize} 
\end{frame}
\appendix
\section*{Appendix} 
\end{document}


