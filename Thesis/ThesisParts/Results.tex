\chapter{Results and Analysis}

After the implementation phase of the design discussed in the previous chapter, the next phase involved performing tests to check if the implementation worked. To recap the problem statement intended to be solved through the thesis was to increase the performance of the cuttlefish application by adding more computational capacity and distributing the objects to utilize the added computational power. The first part of the goal was achieved by creating a distributed system of heterogeneous systems as compute nodes connected through the enterprise network. For utilizing the computational capacity added, the application had to be modified to perform distribution of the work items amongst the computes nodes, collect their partial results and use the partial result to generate the final output. The desired result to be achieved through the thesis primarily was less computation time for multiple models as compared to computation time needed by current cuttlefish application and increase resource utilization by making use of the available machines.\newline

\section{Overview of SDLC Testing Phase}

In the software development life-cycle, testing phase follows the implementation phase wherein various tests are conducted to verify and validate the implemented software. The result of the testing phase depends highly upon the type of test cases which are run. Testing can be performed by manually testing the software or by automating the tests.In manual testing, the tester uses the software as an end-user and makes note of the deviations seen in the behavior of the system as from the expected (if any). In automated testing, another software or scripts are used to run the various test-cases for the software under consideration. Each of these broadly classified types of testing techniques can consists of unit testing, integration testing, system testing, performance testing, regression testing etc.\newline

Manual testing as stated earlier is done by the tester by running the various scenarios in which an end user might use the system in. Mostly, manual testing is done to verify the system's behavior with respect to the specification which involves unit testing, integration testing. The validation of the implementation is with respect to satisfying the needs of the customer which generally involves user acceptance tests and usability tests. Automated testing is generally done to perform regression testing where in part of the software under consideration is being fixed for some bug found by verification process by re-writing a piece of code or introducing the fix by a new piece of code. The goal of regression tests is to ensure that the fixed bug is not leading to new bugs in the system and thus is done quite frequently i.e. each time a discovered bug is fixed. Automation testing is also used to perform stress test or performance tests where in the system is subjected to different loads and the behavior is observed with the primary goal to measure the performance or efficiency of the system under varying loads. As automation testing involves using other software or scripts to perform the tests, it cheaper and efficient both in terms of money and time. \newline 

Unit tests:        

\section{Implementation Assessment}

\section{Results}